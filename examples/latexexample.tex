\section{LaTeX example}

The confy setup block tells confy how meta-instructions will be embedded in 
this file. Confy supports line and block comments, and you should specify at
least one of the two for each of inert text and metacode. 

LaTeX only supports line comments. In general, the sensible approach is to
take the current file's language's comment syntax, and suffix it with some
unique character sequence to avoid accidental clashes with any existing
comments that should not have any meaning to confy.

% confy-setup { line: "%-", meta_line: "%!" }

We can now define some variables. The quoted text is optional, and allows
 you to display more memorable descriptions in interactive mode.

%! bool $b1 "First boolean" = false;
%! bool $b2 "Second boolean" = true;

Are both booleans true?
%! if($b1 && $b2) {
%-Yes!
%! } else {
No!
%! }

If you now were to run confy and set one of the two booleans to false, 
confy would change this file to comment out the line saying `Yes!', and
uncomment the line saying `No!'.

For more advanced usage, you can also use \textit{templating}:

%! int $i1 "First integer" = 42;
%! template {
%-The integer is $i1.
%! } into {
The integer is 42.
%! }

This text will be updated to reflect the chosen value of $i1.

